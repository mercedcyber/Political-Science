\documentclass{article}
\usepackage[utf8]{inputenc}
\usepackage{color}
\title{POLI 101 - Midterm Review 1}
\author{Ramiro Gonzalez} 
\begin{document}

\maketitle
%Web Crawler 
\section*{ Quiz questions}
\begin{enumerate}
    \item Which describes the trend of number of primaries involved each cycle in selecting a president?\\
    \color{red}Number of primaries has gone up\color{black} 
    \item One issue consistently debated over time has been the distribution of power between elites and the electorate. Political parties have not been exempt from this debate (and sometimes controversy). As time has gone on, has the party relinquished more or less control to the electorate?
    \color{red}The electorate has gained more control
\color{black}
    \item Which reason best describes why the electoral college was created?\\
    \color{red} Distrust in the competency of the general public
 \color{black} 
    \item Does a president have more power under unified government or divided government?\\
    \color{red} Unified government
 \color{black} 
    \item If a president does not approve of a law passed by Congress, and Congress is not in session, he/she can essentially ignore the bill instead of explicitly vetoing the bill. What is this called?\\
    \color{red}Pocket veto \color{black} 
    \item What is the main benefit of policy centralization?\\
    \color{red}It protects the president's programs \color{black} 
    \item Which describes the role the president has in judges becoming judges? \\
    \color{red}  The president nominates people to the position \color{black} 
    \item Who can the president appoint as a cabinet secretary? \\
    \color{red} Anyone \color{black} 
    \item What is the main role of the Chief of Staff? \\
    \color{red} Whatever the president wants that role to be \color{black} 
    \item What group is the first to make formal steps in approving/confirming a president's judicial nomination? \\
    \color{red}  Senate Judiciary Committee\color{black} 
    \item Presidents fare better on Supreme Court cases directly after a war ends than they do directly after a war begins.\\
    \color{red}True \color{black}
    \item Howell \& Ahmed (2012) use fixed effects (for Justices) in their empirical analyses. What does this accomplish?\\
    \color{red} Accounts for differences over time under different Justices \color{black}
    \item What authority does the Department of Labor have? (Select any that apply)\color{red}
    \begin{itemize}
        \item  Enforce wage laws
        \item  Implement wage laws
        \item  Enforce labor laws
        \item  Implement labor laws
    \end{itemize}\color{black}
    \item What article and section of the constitution grants authority for the secretary to advise the president?\\
    \color{red}Artile II, Section 2 \color{black} 
\end{enumerate}
\section*{Journal articles (groups 1-4)}
\begin{enumerate}
    \item Group 1 : Unilateral Action and Presidential Power: A Theory\\
    
    \begin{enumerate}
        \item \color{red} Main Conclusion: \color{black} \textbf{Presidents have incentives to expand their institutional power, and they operate within a formal governance structure whose pervasive ambiguities-combined 
with advantages inherent in the executive nature of the presidential job-give them countless opportunities to move unilaterally into new territory, claim new powers, and make
policy on their own authority.}
        \begin{itemize}
            \item The president's formal capacity for taking unilateral action and thus for making law on his own.
            \item Unilateral action has grown increasingly more central to the modern presidency. 
            \item Central Claim: The president's power of unilateral action are a force in American Politics precisely because they are not specified in the formal structure of government. 
            \item Presidential power derives its strength and resilience from the ambiguity of the formal structure. 
            \item The dominant approach was that Strong presidential leadership is due to strong skills, temperament and experience and not  formal powers. "The foundation of presidential power is ultimately personal." 
        \end{itemize}
        \item \color{red} What it means for the president: \color{black} The current president will continue to shift the balance of power in his favor, and due to the ambiguities in the constitution and because the Supreme Court has the incentive to be sympathetic, however the president can only go so far until  Congress or the courts react, therefore the president will move forward slowly. 
    \end{enumerate}
    \color{black}
    \item Group 2: Voting For The President: The Supreme Court During War
    \begin{enumerate}
        \item \color{red} Main Conclusion: \color{black} \textbf{And central to this argument is a conviction that judges predictably uphold elements of presidents’ policy agendas in war that would not withstand judicial scrutiny in peace.}
        \begin{itemize}
            \item  We find evidence not only that presidents fare better on Supreme Court cases in the immediate aftermath of a war’s beginning; but also that,contrary to existing claims about ratchet effects, they also fare worse in the immediate aftermath of a war’s termination
        \end{itemize}
       
        \item \color{red}What it means for the president: \color{black} For president George W. Bush after the 9/11 attack on the world trade center his invasion of iraq and military action againgts Iraq was not vehemently opposed by the judicial system. At the beginning of the iraq war, George W. Bush was not challenged. 
    \end{enumerate}
    \color{black}
    \item Group 3 : Who Influences Whom? The President, Congress, and the Media 
    
    \begin{enumerate}
        \item \color{red} Main Conclusion: \color{black} Findings on the presidents ability to set agenda of Congress and the media is inconclusive. 
Most of the time, all three react to events and issues, even in foreign policy arena. 
        \item \color{red} What it means for the president: \color{black} The current president continues to attack the media and calls congress incompetent, doing so will not lead the media or congress to comply with his demands. 
    \end{enumerate}
    \color{black}
    \item Group 4: Toward a Broader Understanding of Presidential Power: A Reevaluation of the Two Presidencies
Thesis
    
    \begin{enumerate}
        \item \color{red} Main Conclusion: \color{black}
        \begin{itemize}
            \item \textbf{United States has one presidency for domestic matters along  with  a  second,  more  powerful  presidency  for foreign  affairs}
            \item Findings have strong support for the two presidents thesis. 
            \item The coefficient for foreign affairs is significant at conventional levels, With or without president indicators. \\
            A president's change in budgetary appropriations was 8\% closer in foreign and defense agencies than it is in domestic ones. 
            \item The effect is similar in a unified government, showing budgetary success is between 7\% and 10\% greater when the president and congress share
            partisan affiliation.
            \item Additionally their findings showed that the two presidencies effect is lower during periods of bipartisanship. However, even when accounting for this impact, there are still signs of the two presidents effects. \\
            Effect do not disappear when bipartisanship in foreign affairs declines. 
            \item Presidents achieve more budgetary success on foreign policy, and this result holds even when accounting for the possibility of greater bipartisanship in 
            foreign affairs. 
            
        \end{itemize}
        \item \color{red} What it means for the president: \color{black} If it true that the two presidents thesis holds this means the current president will continue to concentrate on foreign affairs more than domestic since a foreign affairs presidency is more powerful, and this presidents loves power. 
    \end{enumerate}
    \color{black}

\end{enumerate}
%\section*{Book Chapters (1-9)}
%\subsection*{Chapter 1}
%\begin{enumerate}
   % \item \textbf{President Centered} 
    %\item \textbf{Institutional based} 
%\end{enumerate}
%\subsection*{Chapter 2}
%\subsection*{Chapter 6}
%\subsection*{Chapter 7}
%\subsection*{Chapter 8}
%\subsection*{Chapter 9}

\section*{Policymaking slides (1-51)}
\begin{enumerate}
    \item \textbf{Interstate Commerce Act - } To address the problem of rate discrimination by 
the railroads against farmers
    \item \textbf{Woodrow Wilson} Expanded presidential role. \begin{itemize} 
        \item Saw the presidency as similar to a prime minister - To  propose an integrated set of measures that addressed social and economic problems and then to use personal  and political influence to see that they get enacted. 
        \item Helped formulate and even draft legislation. 
        \item First president since John Adams to personally deliver state of the union. 
        \item introduced legislation that created the Federal Reserve 
System for the purpose of stabilizing the money system 
    \end{itemize} 
    \item \textbf{Supreme Court} continued to take a  restrictionist view of the expanded role of the federal government 
    \begin{itemize}
        \item Gutted  the FTC Act, the Clayton Act, etc. as they  pertained to anything other than the federal level
        \item estrictive interpretation of the 10th amendment implied that the federal government did not have the 
authority to regulate matters in intrastate commerce or to interfere in the economies or affairs  of the states
    \end{itemize}
        \item  \textbf{ Republicans Harding, Coolidge,  and Herbert Hoover} did not seek to expand the presidents policymaking role. 
        \item \textbf{The depression} did more to expand the president’s role than any other single event.  
        \item \textbf{Frankling Roosevelt} Chief Policymaker 
        \begin{itemize}
            \item New Deal 
            \item Drafted Legislation 
            \item When the Supreme Court struck down major components of the New Deal, Roosevelt threatened to  “pack” the court with one new justice for every one that opposed the program.  
        \end{itemize}
        \item  \textbf{Every president} from Truman through  Clinton has presented to Congress a legislative program.  When Eisenhower in his first year failed to present one, he was soundly criticized so that he  did submit one every other year
\end{enumerate}


\end{document}
