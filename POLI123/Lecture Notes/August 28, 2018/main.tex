\documentclass{article}
\usepackage[utf8]{inputenc}
\usepackage{amsmath}
\usepackage{booktabs}
\usepackage{color}
\usepackage{lipsum}
\usepackage{amsmath}
\usepackage{amssymb}
\usepackage{fancyhdr}
\usepackage[T1]{fontenc}
\usepackage{fourier}
\usepackage{listings}
\usepackage{graphicx}
\usepackage{xcolor}
\graphicspath{ {images/} }
% document size parameters

\setlength{\oddsidemargin}{0.0in}
\setlength{\evensidemargin}{0.0in}
\setlength{\textheight}{8.4in}
\setlength{\textwidth}{6.5in}
\setlength{\voffset}{-0.40in}
\setlength{\headsep}{26pt}
\setlength{\parindent}{0pt}
\setlength{\parskip}{6pt}

% header information
\pagestyle{fancyplain}
\lhead{\large{{\bf POLI 123} }}
\rhead{\large{{\bf Political Psychology}}}
\title{Foundations}
\author{Ramiro Gonzalez }
\date{}
\begin{document}

\maketitle

\section*{Institutions}
\begin{itemize}
    \item What laws, organizations, or customs structure or regulate our behavior? 
    \begin{itemize}
        \item When do institutions, psychology, and/or individual characteristics matter most? 
        \item Why did you show up today? Punishment, Disapproval of peers, Enthusiastic about class? 
        \item Why do people vote? \\
        \textbf{People are expected to vote.  Voting is part of the culture. Self interest such as people voted for slavery even though they did not own slaves, in hopes that someday they can own slaves and move up the ladder}
        \begin{enumerate}
            \item Institution: Being judge for not voting. Groups and organization. Institutions provide certain benefits. 
            \item Psychology: Desire to have a voice. Habit if you have voted in the past you will vote in the future. Social norms (Your friends voted). 
        \end{enumerate}
        \item Why vote at all?
        \begin{itemize}
            \item Is it rational to vote?
            \item If $C < B\cdot P$. 
            \begin{itemize}
                \item B is benefit. Personal Benefit. 
                \item P probability vote matters. (Probability is generally small, millions of people voting.) One vote that tips the scale. 
                \item C your cost. (If cost is larger why show up? ) Is voting irrational if vote does not matter? 
            \end{itemize}
            \item If we add D Civic Duty $$C < B\cdot P + D$$. Is it now rational? 
            \item what about benefit to myself but to millions of people? Benefit increases. 
        \end{itemize}
    \end{itemize}
    \item Local elections matter the most. Why are we not voting in local elections.
    \begin{enumerate}
        \item People overestimate the likelihood that vote will be the tipping point, this is called voter's illusion. 
        \item Two types of emotion, Enthusiasm and anger. Local elections are not as flashy. 
        \item Personality (Certain people are more likely to vote) , Conscientious
        \item How much do we care about others? How much are you willing to give to a stranger? (Altruist votes more than Scrooge). People who are more likely to share are more likely to vote.
        \item Genetics, is related to whether you show up to vote or not. Nurture: a family that votes. Nurture versus nurture. Something in DNA that affects how likely we are to vote. 
        \item Social Norms. What is expected by others? 
        \item Alienation or Inclusion. If one feels government is here to serve vs pushed away by government. 
    \end{enumerate}
    \item Whom do I vote for? 
    \begin{itemize}
        \item Appeals to emotions, prejudice. What is most convenient. 
        \item Party affiliation, loyalty to a group and organization, and you believe your values align with the party. Logistics, party affiliation is a shortcut, voting for a platform. 
        \item Trust/Character (real and perceived)
        \item Popularity (Appearance). Custom built podiums. Taller candidates, and better looking candidates tend to win. Name recognition. 
        \item Consistency: Hillary Clinton has flopped. switching positions.
        \item Incumbents are very likely to win. 
        \item Religion: Major religion is the area such as united states is christian. 
    \end{itemize}
    \item Strategic Voting. Who is most likely to win? 
    \begin{itemize}
        \item In 2000: \\
        Who should Ralph Nader supporters vote for? 
        \item Pick which one stands for the same values as you. Voting to make a statement. \\
        $\frac{1}{2}$ went to gore, $\frac{1}{2}$ stayed with Nader. 
        \item Why did Nader supporters stayed with Nader? 
        \begin{enumerate}
            \item Vote does not matter, making a statement. Long term investment in third party. 
        \end{enumerate}
    \end{itemize}
    \item WHO votes?
    \begin{itemize}
        \item If you are richer you are most likely to vote. Greater resources, less barriers. (Socioeconomic resources)
        \item Attitudes (Efficacy, duty, interest, trust, alienation, indifference) 
        \item Local elections are unrepresentative. 
        \begin{itemize}
            \item Higher education, higher salary, age, and employed means more voting. 
        \end{itemize}
        \item Can psychology help? How level the playing field? 
    \end{itemize}
\end{itemize}
\section*{Social Science can be unrepresented}
\begin{itemize}
    \item Western, Educated, Industrialized, Rich, Democratic (W.E.I.R.D)\\
    No representative of the united states, research was western, and small sample space.\\
    Social science is not well funded. \\
    \item Beyond WEIRD Morality. 
    \item Who is doing the research? Research is influenced by the researcher. 
\end{itemize}
\section*{Parts or a research paper}
\begin{enumerate}
    \item Main Argument.  
    \item How research is conducted. 
    \item Main Results. 
    \item Skeptical, WEIRD, Not representative
    \item Why does it matter, does it apply to the real world. 
\end{enumerate}
\end{document}










